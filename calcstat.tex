\documentclass[12pt,letterpaper]{article}
\usepackage[utf8]{inputenc}
\usepackage{amsmath}
\usepackage{amsfonts}
\usepackage{amssymb}
\usepackage{amsthm}
% \usepackage{textcomp}
% \usepackage{graphicx}
% \usepackage{lmodern}
% \usepackage[left=2cm,right=2cm,top=2cm,bottom=2cm]{geometry}
\title{The Calculus of Statistics}
\author{Blaise Whitesell \and Jeremiah Gelb}
\date{\today}
\begin{document}
% Begin hack commands
\newcommand{\bdef}[1]{\textbf{#1}} % Bold definitions
\newcommand{\dx}{\:\mathrm{d}x} % Integral dx
\newcommand{\dt}{\:\mathrm{d}t} % Integral dt
\newcommand{\dth}{\:\mathrm{d}\theta} % Integral dtheta
\newtheorem{problem}{Problem}
% End hack commands
\maketitle
\section{Probability Distribution Functions}
\subsection{Probability Density Functions}
A probability density function (\bdef{PDF}) describes
the relative likelihood $f(x)$ of possible outcomes
for a continuous random variable.
\begin{itemize}
\item The probability of any $x$ occurring is either positive or zero,
so a PDF can never be negative.
\begin{equation*}
f(x) \geq 0 \quad \text{for all }x
\end{equation*}
\item The area under the curve must equal 1
\begin{equation*}
\int_a^b f(x)\dx = 1
\end{equation*}
where $a$ and $b$ are the lower and upper bounds,
often $-\infty$ and $\infty$.
\end{itemize}
\subsection{Cumulative Distribution Functions}
A cumulative distribution function (\bdef{CDF}) gives
the probability $F(x)$ that the outcome of a
continuous random variable will be less than or equal to $x$.
\begin{itemize}
\item The CDF is related to the PDF by this integral:
\begin{equation*}
F(x) = \int_a^x f(t) \dt \quad \text{for } a \leq x \leq b
\end{equation*}
% where $F(x)$ is the CDF and $f(x)$ is the PDF.
\item A CDF is monotonically increasing on the interval $(a,b)$.
\begin{equation*}
F(a) = 0 \qquad F(b) = 1
\end{equation*}
\end{itemize}
\subsection{Using Calculus}
To find the probability of an outcome in a certain range,
one can integrate the PDF over that interval
$(c,d)$ contained in $(a,b)$.
\begin{equation*}
\int_c^d f(x) \dx \quad = F(d) - F(c)
\end{equation*}
This gives the area under the curve, corresponding to the probability.
\section{The Uniform Distribution}
\subsection{Definition}
A uniform distribution describes a continuous random variable
where every outcome on an interval $(a,b)$ is equally likely.
\begin{equation*}
f(x) = \kappa
\end{equation*}
\subsection{Implications}
\begin{itemize}
\item The distribution looks like a rectangle
with length $b-a$ and height $\kappa$
\item Area $= 1 = \kappa (b-a)$
% Gratuitous calculus follows:
\begin{equation*}
\int_a^b \frac{1}{b-a} \dx = \frac{x}{b-a}\bigg|_a^b
= \frac{b}{b-a} - \frac{a}{b-a} = \frac{b-a}{b-a} = 1
\end{equation*}
\item Therefore, we have an explicit definition for $\kappa$:
\begin{equation*}
\kappa = \frac{1}{b-a}
\end{equation*}
\end{itemize}
\subsection{PDF of a uniform distribution}
The formal equation is piecewise.
It depends only on the bounds $a$ and $b$.
\begin{equation*}
f(x)=
\begin{cases}
0 & \text{if } x < a,\\
1/(b-a) & \text{if } a \leq x \leq b,\\
0 & \text{if } x > b.
\end{cases}
\end{equation*}
\subsection{CDF of a uniform distribution}
The CDF increases linearly on the interval $(a,b)$.
\begin{equation*}
f(x)=
\begin{cases}
0 & \text{if } x < a,\\
(x-a)/(b-a) & \text{if } a \leq x \leq b,\\
1 & \text{if } x > b.
\end{cases}
\end{equation*}
\begin{problem}[Spin to Win]
What is the probability of losing on the first spin?
\begin{proof}[Solution]
There are 200 spaces,
exactly one of which corresponds to losing immediately.
The range of values can be modeled as a uniform distribution from 0$^\circ$ to 360$^\circ$.
\begin{equation*}
1/200 \times 360^\circ = 1.8^\circ
\end{equation*}
\begin{equation}
\int_0^{1.8} \frac{1}{360} \dth
= \frac{\theta}{360} \bigg|_0^{1.8}  = \frac{1.8}{360} - \frac{0}{360} = 0.005
\end{equation}
Interestingly enough, this is equivalent to $1/200$.
\end{proof}
\end{problem}


\end{document}
