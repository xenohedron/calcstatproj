\documentclass[11pt,leqno,letterpaper]{article}
\usepackage[utf8]{inputenc}
\usepackage{amsmath}
\usepackage{amsfonts}
\usepackage{amssymb}
\usepackage{amsthm}
% \usepackage{textcomp}
% \usepackage{graphicx}
% \usepackage{lmodern}
\usepackage[left=4cm,right=4cm,top=4cm,bottom=4cm]{geometry}
\title{\Huge{The Calculus of Statistics}}
\author{\Large{Blaise Whitesell} \and \Large{Jeremiah Gelb}}
% \date{\today}
\begin{document}
% Begin hack commands
\newcommand{\bdef}[1]{\textbf{#1}} % Bold definitions
\newcommand{\dx}{\:\mathrm{d}x} % Integral dx
\newcommand{\dt}{\:\mathrm{d}t} % Integral dt
\newcommand{\dth}{\:\mathrm{d}\theta} % Integral dtheta
\newcommand{\ddt}{\frac{\mathrm{d}}{\mathrm{d}t}} % d/dt
\theoremstyle{definition} \newtheorem{problem}{Problem}
\newcommand{\psep}{\hrule \vspace{1 ex}} % Separator line for problems
\newcommand{\AM}{\textsc{am}}
% End hack commands
\maketitle
\vspace{2 em}
\section{Probability Distribution Functions}
\subsection{Probability Density Functions}
A probability density function (\bdef{PDF}) describes
the relative likelihood $f(x)$ of possible outcomes
for a continuous random variable.
\begin{itemize}
\item The probability of any $x$ occurring is either positive or zero,
so a PDF can never be negative.
\[
f(x) \geq 0 \quad \text{for all }x
\]
\item The area under the curve must equal 1
\[
\int_a^b f(x)\dx = 1
\]
where $a$ and $b$ are the lower and upper bounds,
often $-\infty$ and $\infty$.
\end{itemize}
\subsection{Cumulative Distribution Functions}
A cumulative distribution function (\bdef{CDF}) gives
the probability $F(x)$ that the outcome of a
continuous random variable will be less than or equal to $x$.
\begin{itemize}
\item The CDF is related to the PDF by this integral:
\[
F(x) = \int_a^x f(t) \dt \quad \text{for } a \leq x \leq b
\]
% where $F(x)$ is the CDF and $f(x)$ is the PDF.
\item A CDF is monotonically increasing on the interval $(a,b)$.
\[
F(a) = 0 \qquad F(b) = 1
\]
\end{itemize}
\subsection{Using Calculus}
To find the probability of an outcome in a certain range,
one can integrate the PDF over that interval
$(c,d)$ contained in $(a,b)$.
\[
\int_c^d f(x) \dx \quad = F(d) - F(c)
\]
This gives the area under the curve, corresponding to the probability.
\section{The Uniform Distribution}
\subsection{Definition}
A uniform distribution describes a continuous random variable
where every outcome on an interval $(a,b)$ is equally likely.
\[
f(x) = \kappa
\]
\subsection{Implications}
\begin{itemize}
\item The distribution looks like a rectangle
with length $b-a$ and height $\kappa$
\item Area $= 1 = \kappa (b-a)$
% Gratuitous calculus follows:
\[
\int_a^b \frac{1}{b-a} \dx = \frac{x}{b-a}\bigg|_a^b
= \frac{b}{b-a} - \frac{a}{b-a} = \frac{b-a}{b-a} = 1
\]
\item Therefore, we have an explicit definition for $\kappa$:
\[
\kappa = \frac{1}{b-a}
\]
\end{itemize}
\subsection{PDF of a uniform distribution}
The formal equation is piecewise.
It depends only on the bounds $a$ and $b$.
\[
f(x)=
\begin{cases}
0 & \text{if } x < a,\\
1/(b-a) & \text{if } a \leq x \leq b,\\
0 & \text{if } x > b.
\end{cases}
\]
\subsection{CDF of a uniform distribution}
The CDF increases linearly on the interval $(a,b)$.
\[
f(x)=
\begin{cases}
0 & \text{if } x < a,\\
(x-a)/(b-a) & \text{if } a \leq x \leq b,\\
1 & \text{if } x > b.
\end{cases}
\]
\psep
\begin{problem}
In the game ``Spin to Win'', the player spins a giant roulette wheel
with 200 spaces. One is marked ``Win'', one is marked ``Lose'',
and the remaining 198 are marked ``Spin Again''.
What is the probability of losing on the first spin?
\begin{proof}[Solution]
There are 200 spaces,
exactly one of which corresponds to losing immediately.
The continuous range of values can be modeled as
a uniform distribution from 0$^\circ$ to 360$^\circ$.
\[
1/200 \times 360^\circ = 1.8^\circ
\]
\[
\int_0^{1.8} \frac{1}{360} \dth
= \frac{\theta}{360} \bigg|_0^{1.8} 
= \frac{1.8}{360} - \frac{0}{360} = 0.005
\]
Interestingly enough, this is equivalent to $1/200$.
\end{proof}
\end{problem}
\psep
\section{The Exponential Distribution}
\subsection{Introduction}
An exponential distribution can be used to model events that occur
independently at a constant average rate.\\
\psep
\begin{problem}
250 kids at a frat party are randomly getting sick at a continuous
rate of 20\% per hour, beginning at 12:00\AM .
What are the odds that Chad vomits between 2:00\AM{ }and 3:00\AM ?
\begin{proof}[Solution]
Let $t=0$ at 12:00\AM .
Then the number of kids who are not sick is given by $250e^{-.2t}$.
This means that the number of kids who are sick is given by
$250 - 250e^{-.2t}$.
Therefore, the proportion of kids who are sick after $t$ hours is
\[
\frac{250 - 250e^{-.2t}}{250} \quad = 1 - e^{-.2t}
\]
This is equivalent to the probability of being sick after $t$ hours,
which is the CDF. To find the PDF, simply take a derivative.
\begin{align*}
\int_0^t f(x) \dx &= 1 - e^{-.2t}\\
\ddt \int_0^t f(x) \dx &= \ddt (1 - e^{-.2t})\\
f(x) &= .2e^{-.2t}
\end{align*}
Now that we know the PDF, we integrate it between $t=2$ and $t=3$.
\[
\int_2^3 .2e^{-.2t} = -e^{-.2t}\Big|_2^3 = -e^{-.6} - (-e^{-.4})
= .1215
\]
Note that we could also have found this answer by
finding the difference between $F(3)$ and $F(2)$,
without using the PDF at all.
\[
\left(1 - e^{-.2t}\Big|_{t=3}\right) -
\left(1 - e^{-.2t}\Big|_{t=2}\right) = .1215
\]
The probability that Chad will vomit between 2:00\AM{} and 3:00\AM{}
is $.1215$.
\end{proof}
\end{problem}
\psep
\section{The Beta Distribution}
\section{The Normal Distribution}

\end{document}
