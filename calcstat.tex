\documentclass[12pt,letterpaper]{article}
\usepackage[utf8]{inputenc}
\usepackage{amsmath}
\usepackage{amsfonts}
\usepackage{amssymb}
% \usepackage{graphicx}
% \usepackage{lmodern}
% \usepackage[left=2cm,right=2cm,top=2cm,bottom=2cm]{geometry}
\title{The Calculus of Statistics%: Continuous Probability Distributions
}
\author{Blaise Whitesell \and Jeremiah Gelb}
\date{\today}
\begin{document}
% Begin hack commands
\newcommand{\bdef}[1]{\textbf{#1}} % Bold definitions
\newcommand{\dx}{\:\mathrm{d}x} % Integral dx
% End hack commands
\maketitle
\section{Probability Distribution Functions}
\subsection{Probability Density Functions}
A probability density function (\bdef{PDF}) describes the relative likelihood $f(x)$ of possible outcomes for a continuous random variable.
\begin{itemize}
\item The probability of any $x$ occurring is either positive or zero, so a PDF can never be negative.
\begin{equation*}
f(x) \geq 0 \text{ for all }x
\end{equation*}
\item The area under the curve must equal 1
\begin{equation*}
\int_a^b f(x)\dx = 1
\end{equation*}
where $a$ and $b$ are the lower and upper bounds, often $-\infty$ and $\infty$.
\end{itemize}
\subsection{Cumulative Distribution Functions}
A cumulative distribution function (\bdef{CDF}) gives the probability $F(x)$ that the outcome of a continuous random variable will be less than or equal to $x$.
\begin{itemize}
\item A CDF is monotonically increasing on the interval ($a$,$b$).
\begin{equation*}
F(a) = 0 \text{ ; \quad}F(b) = 1
\end{equation*}
\end{itemize}
\subsection{Calculus (placeholder)}
%integral of pdf gives cdf
\section{The Uniform Density Function}
\subsection{Definition}
A uniform density function describes a continuous random variable where every outcome on an interval ($a$,$b$) is equally likely.
\begin{equation*}
f(x) = \kappa
\end{equation*}
\subsection{Implications}
\begin{itemize}
\item The function looks like a rectangle with length $(b-a)$ and height $\kappa$
\item Area $= 1 = (b-a)*\kappa$
\item Therefore, we have an explicit definition for $\kappa$:
\begin{equation*}
\kappa = \frac{1}{b-a}
\end{equation*}
\end{itemize}
\subsection{Equation of a uniform density function}
The formal equation is piecewise. It depends only on the bounds $a$ and $b$.
\begin{equation*}
f(x)=
\begin{cases}
0 & \text{if } x < a,\\
\frac{1}{b-a} & \text{if } a \leq x \leq b,\\
0 & \text{if } x > b.
\end{cases}
\end{equation*}







\end{document}
